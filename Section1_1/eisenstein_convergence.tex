
\documentclass {amsart}
\usepackage{amsmath, amsthm, amsfonts, amssymb}
%\usepackage{tikz} \usetikzlibrary{positioning} % for commutative diagrams

%%%% Theorem-type MACROS

\theoremstyle{plain}

\newtheorem{proposition}{Proposition}
\newtheorem{theorem}[proposition]{Theorem}
\newtheorem{corollary}[proposition]{Corollary}
\newtheorem*{observe}{Observation}
\newtheorem{lemma}[proposition]{Lemma}

\theoremstyle{definition}
\newtheorem{definition}{Definition} 
\newtheorem{exercise}{Exercise} 
\newtheorem{axiom}{Axiom}
\newtheorem{declaration}[axiom]{Declaration}

\theoremstyle{remark}
\newtheorem{example}{Example}
\newtheorem*{remark}{Remark}
\newtheorem*{note}{Note}
\newtheorem*{assumption}{Background Assumption}


%%%% Character MACROS
\newcommand{\defeq}{\stackrel{\mathrm{def}}{\, = \,}}

\newcommand{\bN}{{\mathbb{N}}}
\newcommand{\bZ}{{\mathbb{Z}}}
\newcommand{\bR}{{\mathbb{R}}}
\newcommand{\bC}{{\mathbb{C}}}
\newcommand{\bQ}{{\mathbb{Q}}}
\newcommand{\bF}{{\mathbb{F}}}
\newcommand{\bA}{{\mathbb{A}}}
\newcommand{\bP}{{\mathbb{P}}}

\newcommand{\h}{{\mathcal{H}}}





%%%%%%%%%%  END OF MACROS  %%%%%%%%%%%%

\begin{document} 

\title{Eisenstein Series}

\author{Modular Form Study Group}

\date{August 2014, modified \today}

\maketitle

%%%%%%%%%%%%%%%%%%%%

\section {Introduction}

In this document we consider the convergence of
the Eisenstein Series $G_k$ for $k \ge 4$.

%%%%%%%%%%%%%%%%%%%%

\section {Estimates on $|a z - b|$}

Let $a, b \in \bR$ be fixed. 
Likewise, fix positive $X, Y \in \bR$.
We consider $z$ in the set 
$$S \; \defeq \; \{ z = x + y i \in \bC \mid \hbox{$|x| \le X$  and $y \ge Y$} \}.$$

\begin{lemma}
There is a constant $C_1 > 0$, independent of $a, b$, such that, for all $z\in S$,
$$
|a z + b| \ge C_1 |b|.
$$
\end{lemma}

\begin{proof}
Fix $\lambda \in \bR$ such that $0 < \lambda < 1/X$.
If $|a| \ge \lambda |b|$ then we use the imaginary part of $a z + b$ as a lower bound, which
is $a y$. Thus
$$
|a z + b| \ge |a y|  \ge |a| Y \ge (\lambda Y) |b|.
$$

On the other hand, if $|a| \le \lambda |b|$ then we use the real part of $a z + b$,
which is $a x + b$, to obtain a lower bound. Thus
$$
|a z + b| \ge |a x + b|  \ge |b| - |ax| \ge |b| - |b| (\lambda X) = (1 - \lambda X) |b| 
$$

The result follows by setting $C_1$ equal to the minimum of $\lambda Y$ and $1 - \lambda X$.
\end{proof}

\begin{remark}
 The optimal choice of $\lambda$ in the proof is $1/(X + Y)$ in which case we have $C_1 = Y/(X+Y)$. However, the proof seems clearest when we use a more  general~$\lambda$.
\end{remark}

\begin{lemma}
There is a constant $C > 0$, independent of $a, b$, such that, for all $z\in S$,
$$
|a z + b| \ge C \max( |a|,  |b|).
$$
\end{lemma}

\begin{proof}
The imaginary part of $a z + b$ is $a y$. Thus
$$
|a z + b| \ge |a y|  \ge Y |a|.
$$
Let $C$ be the minimum of $Y$ and $C_1$, where $C_1$ is as in the previous lemma.
So $|a z + b| \ge C |a|$ and $|a z + b| \ge C |b|$. The result follows.
\end{proof}


%%%%%%%%%%%%%%%%%%%%

\section {A Key sum}

Now fix a positive integer~$k \ge 3$ and consider the sum
$$
A = \sum \ \!\! '  \; \max(|a|, |b|)^{-k}
$$
where the sum is over all pairs of integers $(a, b)$ not equal to $(0, 0)$.
By a simple counting argument,  for all positive integers~$n$ the number of pairs less 
with $\max (a, b) = n$ is equal to $8 n$.
So
$$
A =  \sum_{n=1}^\infty (8 n) \cdot n^{-k} =
8  \sum_{n=1}^\infty \frac{1}{n^{k-1}} =  8 \zeta (k-1).
$$
In particular, $A < \infty$.


%%%%%%%%%%%%%%%%%%%%

\section {Eisenstein Convergence}

Fix positive $X, Y \in \bR$. As above, let $S$ be as follows
$$S \; \defeq \; \{ z = x + y i \in \bC \mid \hbox{$|x| \le X$  and $y \ge Y$} \}.$$
Fix $C$ so that $|a z + b| \ge C \max(|a|, |b|)$
for all $z \in S$. 
Then, for all $z \in S$,
$$
 \sum \ \!\! '  \frac{1}{|a z + b|^k} \le  \sum \ \!\! ' \frac{1}{(C \max(|a|, |b|))^k}
= \frac{1}{C^k}   \sum \ \!\! ' \frac{1}{(\max(|a|, |b|)^k}
$$
where, as before,  the sum is over all integer pairs $(a, b)$ not equal to~$(0,0)$.
From the above estimate, this sum converges and
$$
 \sum \ \!\! '  \frac{1}{|a z + b|^k} \le \frac{8}{C^k} \zeta (k - 1).
$$

In particular, the Eisenstein sum
$$
G_k (z) =  \sum \ \!\! '  \frac{1}{(a z + b)^k}
$$
is absolutely and uniformly convergent on~$S$. Since 
the upper half plane $\h$ is covered by sets of the form~$S$, this implies that the Eisenstein sum is absolutely convergent for all $z\in \h$, and defines a holomorphic function on~$\h$.


\end{document}
