\documentclass{article}

\usepackage{amsmath, amsthm, amssymb}
%\usepackage{mathrsfs}
%\usepackage[all]{xy}
%\usepackage{enumerate}
%\usepackage[colorlinks]{hyperref}
%\usepackage{cleveref}

\theoremstyle{plain}
\newtheorem{theorem}{Theorem}[section]
\newtheorem{corollary}[theorem]{Corollary}
\newtheorem{proposition}[theorem]{Proposition}
\newtheorem{lemma}[theorem]{Lemma}
\newtheorem{conjecture}[theorem]{Conjecture}

\theoremstyle{definition} 
\newtheorem{definition}[theorem]{Definition}

\theoremstyle{remark}
\newtheorem{remark}[theorem]{Remark} 
\newtheorem{example}[theorem]{Example}


\renewcommand{\emptyset}{\varnothing}
\newcommand{\too}{\longrightarrow}
\renewcommand{\phi}{\varphi}
\newcommand{\eps}{\varepsilon}
\newcommand{\Z}{\mathbb{Z}}
\newcommand{\Q}{\mathbb{Q}}
\newcommand{\R}{\mathbb{R}}
\newcommand{\C}{\mathbb{C}}

\newcommand{\h}{{\mathcal{H}}}

\newcommand{\defeq}{\stackrel{\mathrm{def}}{\, = \,}}

%%%%%%%%%%  END OF MACROS  %%%%%%%%%%%%

\begin{document} 

\title{Section 1.1: Modular forms for $SL_2(\Z)$.}

\author{Modular Form Study Group}

\date{September 2014, modified \today}

\maketitle

%%%

\section{The Modular Group}

The \emph{modular group} is defined to be~$SL_2(\Z)$.
It is generated by the two matrices
$$
T =
\begin{bmatrix}
1 & 1 \\
0 & 1
\end{bmatrix} 
\qquad
S = \begin{bmatrix}
0 & -1 \\
1 & 0
\end{bmatrix} 
$$

\begin{remark}
The matrix $S$ has order~4, and its image has order 2 in $PSL_2(\Z)$.
The matrix $T$ has infinite order.

To show that $S$ and $T$ generate (Exercise~1.1.1), let $G$ be the group generated by~$S$
and $T$. 
Given a coset $\alpha G$ with $\alpha \in SL_2(\Z)$, we show that
we can find a sequence of representatives of the coset 
$$
\alpha = \begin{bmatrix}
a & b \\
c & d
\end{bmatrix} 
$$
with $(c, d)$ decreasing. Eventually wet get a representative with $c = 0$
and $d = 1$, which is a power of~$T$, so is in~$G$. So $\alpha G = G$.
\end{remark}

%%%

\section{Action on $\h$.}

Let $\h$ be the upper half plane.

\end{document}

