
\documentclass {amsart}
\usepackage{amsmath, amsthm, amsfonts, amssymb}
%\usepackage{tikz} \usetikzlibrary{positioning} % for commutative diagrams

%%%% Theorem-type MACROS

\theoremstyle{plain}

\newtheorem{proposition}{Proposition}
\newtheorem{theorem}[proposition]{Theorem}
\newtheorem{corollary}[proposition]{Corollary}
\newtheorem*{observe}{Observation}
\newtheorem{lemma}[proposition]{Lemma}

\theoremstyle{definition}
\newtheorem{definition}{Definition} 
\newtheorem{exercise}{Exercise} 
\newtheorem{axiom}{Axiom}
\newtheorem{declaration}[axiom]{Declaration}

\theoremstyle{remark}
\newtheorem{example}{Example}
\newtheorem*{remark}{Remark}
\newtheorem*{note}{Note}
\newtheorem*{assumption}{Background Assumption}


%%%% Character MACROS
\newcommand{\defeq}{\stackrel{\mathrm{def}}{\, = \,}}

\newcommand{\bN}{{\mathbb{N}}}
\newcommand{\bZ}{{\mathbb{Z}}}
\newcommand{\bR}{{\mathbb{R}}}
\newcommand{\bC}{{\mathbb{C}}}
\newcommand{\bQ}{{\mathbb{Q}}}
\newcommand{\bF}{{\mathbb{F}}}
\newcommand{\bA}{{\mathbb{A}}}
\newcommand{\bP}{{\mathbb{P}}}

\newcommand{\h}{{\mathcal{H}}}





%%%%%%%%%%  END OF MACROS  %%%%%%%%%%%%

\begin{document} 

\title{The Eisenstein Series $G_2$}

\author{Modular Form Study Group}

\date{September 2014, modified \today}

\maketitle

%%%%%%%%%%%%%%%%%%%%

This handout discusses properties of the weight 2 Eisenstein series~$G_2$
defined in Section~1.2 (page 18) of Diamond-Shurman. It is essentially Exercise 1.2.8 (page 23)
of Diamond-Shurman.

%%%%%%%%%%%%%%%%%%%%

\section{An amusing example of conditional convergence}

Consider the sum
$$
\sum_{d\in \bZ} 
\left( \frac{1}{z + d} - \frac{1}{z + d+ 1} \right) = \sum_{d\in \bZ} \frac{1}{(z+d)(z + d + 1)}
$$
where $z \in \bC$ is not an integer.

This sum is absolutely convergent (so we do not need to specify the order of summation).
This follows from the estimate
$$
\sum_{\textit{$d \in \bZ$}} \frac{1}{|( z + d)( z + d + 1)|} \le 
\sum_{\textit{$d \in \bZ$}} \left( \frac{1}{|z + d|^2} +  \frac{1}{|(z + 1) + d|^2}  \right)
$$
Note: we showed the absolute convergence of $\sum (z + d)^{-2}$ in a previous
handout (see \texttt{periodic\_functions.tex}).

This absolutely convergent sum is telescoping:
$$
\sum_{d\in \bZ} 
\left( \frac{1}{z + d} - \frac{1}{z + d+ 1} \right) = 0.
$$

From these considerations we conclude that for $z \in\bC$ not rational
$$
\sum_{c \ne 0} \sum_{d\in\bZ} \left( \frac{1}{c z + d} - \frac{1}{c z + d + 1} \right)
=
0
$$
(where each sum individually is absolutely convergent, so there is no need
to specify order of summation).

Now we reverse the order of summation.  Consider
$$
\sum_{d\in\bZ} \sum_{c \ne 0}  \left( \frac{1}{c z + d} - \frac{1}{c z + d + 1} \right)
=
\sum_{d\in\bZ} \sum_{c \ne 0}  \frac{1}{(c z + d)(c z + d + 1)} 
$$
where $z\in \bC$ is irrational.
We establish the absolute convergence of the inner sum
using the estimate
$$
 \sum_{c \ne 0}  \frac{1}{|(c z + d)(c z + d + 1)|} 
 \le
 \sum_{c \in\bZ}  \frac{1}{|c z+d |^2} +  \frac{1}{|c z + d + 1|^2}.
$$
Note that
$$
 \sum_{c \in\bZ}  \frac{1}{|c z+d |^2} = \frac{1}{|z|^2}
  \sum_{c \in\bZ}  \frac{1}{|(d/z) + c |^2} 
$$
and
$$
 \sum_{c \in\bZ}  \frac{1}{|c z+d + 1 |^2} = \frac{1}{|z|^2}
  \sum_{c \in\bZ}  \frac{1}{|(d + 1)/z + c |^2}.
$$
So both sums are absolutely convergence  (see 
\texttt{periodic\_functions.tex} for the absolute convergence
of sums of the form~$\sum (w + d)^{-2}$).

So the inner sum is absolutely convergent.
We will not worry about the absolutely convergence of the outer sum.
Instead we will consider the summation as given by the limit
$$
\lim_{N \to \infty}
\sum_{d = -N}^N \sum_{c \ne 0}  \left( \frac{1}{c z + d} - \frac{1}{c z + d + 1} \right).
$$
The limited double sum (for fixed $N$) is absolutely convergent, and
by changing the order of summation we get
$$
\sum_{d = -N}^N \sum_{c \ne 0}  \left( \frac{1}{c z + d} - \frac{1}{c z + d + 1} \right)
=
 \sum_{c \ne 0}  \left( \frac{1}{c z  - N} - \frac{1}{c z + N + 1} \right).
$$
From the periodic functions handout (\texttt{periodic\_functions.tex})
we have
$$
\pi \cot \pi z = \lim_{M \to \infty} \sum_{a = -M}^{M} \frac{1}{z - a}.
$$
So if we take the above sum over $c \ne 0$ as given by the limit of
the sum from $c = -M$ to $c=M$, we can use the above cotangent identity to conclude
that
$$
 \sum_{c \ne 0}  \left( \frac{1}{c z  - N} - \frac{1}{c z + N + 1} \right)
 =
\frac{\pi}{z} \cot (- \pi N/z ) + \frac{\pi}{z} \cot ( - \pi (N+ 1)/z )  + \frac{1}{N} + \frac{1}{N+1}.
$$
Recall that 
$$
\cot \pi z = i \, \frac{q + 1}{q - 1}
$$
where $q = e^{2 \pi i z}$ (see \texttt{periodic\_functions.tex}).
This shows that if  $z \in \h$, then 
$$
\lim_{N \to \infty}
\sum_{d = -N}^N \sum_{c \ne 0}  \left( \frac{1}{c z + d} - \frac{1}{c z + d + 1} \right)
= - \frac{2 \pi i} {z}.
$$
We conclude that for $z \in \h$, the given double sum
$$
\sum_{c \ne 0} \sum_{d\in\bZ} \left( \frac{1}{c z + d} - \frac{1}{c z + d + 1} \right)
$$
cannot be absolutely convergent.
(A similar argument can be given for $-z \in \h$).

\begin{remark}
Later we will see that both the inner and outer summations of
$$\sum_{d\in\bZ} \sum_{c \ne 0} \left( \frac{1}{c z + d} - \frac{1}{c z + d + 1} \right)$$
are absolutely convergent.
\end{remark}

%%%%%%%%%%%%%%%%%%%%

\section {The function $G_2$}

The weight two Eisenstein function $G_2: \h \to \bC$ is defined by the formula
$$
G_2 (z) \; \defeq \; \sum_{c \in \bZ} \; \sum_{d \in \bZ}  \,'\, \frac{1}{(c z + d)^2}
= 2 \zeta (2) + \sum_{c \ne 0} \; \sum_{d \in \bZ} \, \frac{1}{(c z + d)^2}.
$$
The prime on the first inner sum means to exclude $d = 0$ if $c =0$.

In another handout (\texttt{periodic\_functions.tex}), we found that for $c \ne 0$ the inner sum
is absolutely convergent and defines a holomorphic periodic function on~$\h$ (and meromorphic on~$\bC$). In fact,
$$
\sum_{d \in \bZ}  \frac{1}{(c z + d)^2} = \frac{\pi^2}{\sin^2 (c \pi z)} = - \frac{4 \pi^2 q^c}{(q^c - 1)^2}
$$
where $q = e^{2 \pi i z}$. Using the well-known power series expansion of $1/(z-1)^2$ (obtained by differentiating the geometric sum expansion of $1/(1-z)$) we get the expansion
$$
\sum_{d \in \bZ}  \frac{1}{(c z + d)^2} =  - 4 \pi^2 \sum_{n=1}^\infty n q^{c n},
$$
where the right hand sum is absolutely convergent for all $z \in \h$.

Since
$$
\sum_{c = 1}^\infty \sum_{n=1}^\infty n |q|^{cn} = \sum_{m = 1}^\infty \sigma(m) |q|^{m}
\le 
\sum_{m = 1}^\infty m^2 |q|^{m}
$$
we get absolute convergence of the double sum
$
\sum \sum n q^{c n}.
$
The outer sum of this sum is essentially the outer sum defining~$G_2(z)$.
So we get the absolute convergence of the outer sum defining $G_2(z)$
for all $z \in \h$, and can derive the $q$-expansion
$$
G_2 (z) 
= 2 \zeta (2) - 8\pi^2  \sum_{m = 1}^\infty \sigma(m) q^{m}.
$$

In summary, \emph{$G_2$ is a periodic holomorphic function
with a nice $q$-expansion. Its double series definition is only conditionally convergent (as
we will see), but individually the inner sum is absolutely convergent, and the outer
sum is absolutely convergent.}

%%%%%%%%%%%%%%%%%%%%

\section {Transformation properties}

Now we derive a double sum expression for~$z^{-2} G_2(-1/z)$.

Recall that the double sum definition of $G_2(z)$ is convergent for $z\in\h$, and
individually the inner and outer sums are absolutely convergent.
Since for all $z\in\h$ we have $-1/z \in \h$, both the inner and outer
sums are absolutely convergent  in the following:
$$
z^{-2} G_2 (-1/z) = \frac{1}{z^2} \sum_{c \in \bZ} \; \sum_{d \in \bZ}  \,'\, \frac{1}{(- c z^{-1} + d)^2}
=
\sum_{c \in \bZ} \; \sum_{d \in \bZ}  \,'\, \frac{1}{(- c + d z)^2}
$$
where the prime in the inner sum means remove the $d=0$ term
if $c \ne 0$.
We rewrite this as
$$
z^{-2} G_2 (-1/z) = 
\sum_{d \in \bZ} \; \sum_{c \in \bZ}  \,'\, \frac{1}{(c z + d)^2}
=
2 \zeta(2) + \sum_{d \in \bZ} \; \sum_{c \ne 0}  \, \frac{1}{(c z + d)^2} 
.
$$
(the last step uses the fact, applied to the outer sum, that the sum of absolutely
convergent series is an absolutely convergent series. In particular, 
both the inner and outer
sums of $ \sum_{d \in \bZ} \; \sum_{c \ne 0}  {(c z + d)^{-2}}$
are absolutely convergent).

%%%%%%%%%%%%%%%%%%%%

\section {The Absolute Convergence Trick}

The $q$-expansion of $G_2$ gives one absolutely convergent expression for $G_2$,
but there is another way to get an absolutely convergent expression involving 
a double sum $\sum_{c \ne 0} \; \sum_{d \in \bZ}$ as in the original definition.

We know that 
$$\sum_{c \ne 0} \; \sum_{d \in \bZ} \, \frac{1}{(c z + d)^2}$$
is absolutely convergent with respect to the first and second summations.
We know that the same is true of
$$
\sum_{c \ne 0} \sum_{d\in\bZ}  \frac{1}{(c z + d) (c z + d + 1)}
$$
and that its value is~$0$ because the inner sum is telescoping.
So we can subtract the two series and simplify as follows

\begin{eqnarray*}
\sum_{c \ne 0} \; \sum_{d \in \bZ} \, \frac{1}{(c z + d)^2}
&=&
\sum_{c \ne 0} \; \sum_{d \in \bZ} \, \frac{1}{(c z + d)^2} -
    \sum_{c \ne 0} \sum_{d\in\bZ}  \frac{1}{(c z + d) (c z + d + 1)}
\\
&=&
\sum_{c \ne 0} \; \sum_{d \in \bZ} \left( \frac{1}{(c z + d)^2}
-
 \frac{1}{(c z + d) (c z + d + 1)}
\right)
\\
&=&
\sum_{c \ne 0} \; \sum_{d \in \bZ}
 \frac{1}{(c z + d)^2 (c z + d + 1)}.
\end{eqnarray*}

This last sum is actually absolutely convergent for $z \in \h$.
This can be seen by using the estimate
$$
\sum_{c \ne 0} \; \sum_{d \in \bZ}
 \frac{1}{|(c z + d)^2 (c z + d + 1)|}
\le
\sum_{c \ne 0} \; \sum_{d \in \bZ}
\left(
 \frac{1}{|c z + d|^3}
 +
 \frac{1}{|c z + d + 1|^3}
 \right).
$$
The right hand side is convergent for all $z\in\h$. This was established
in the Eisenstein Series Handout (\texttt{eisenstein\_convergence.tex}).

Due to absolute convergence we can write
\begin{eqnarray*}
\sum_{c \ne 0} \; \sum_{d \in \bZ} \, \frac{1}{(c z + d)^2}
&=&
\sum_{c \ne 0} \; \sum_{d \in \bZ}
 \frac{1}{(c z + d)^2 (c z + d + 1)}
\\
&=&
\sum_{d \in \bZ}  \; \sum_{c \ne 0}
 \frac{1}{(c z + d)^2 (c z + d + 1)}
\\
&=&
\sum_{d \in \bZ}  \; \sum_{c \ne 0}\left( \frac{1}{(c z + d)^2}
-
 \frac{1}{(c z + d) (c z + d + 1)}
\right)
\end{eqnarray*}
In the previous section we noted that both the inner and outer summations
of
$$
\sum_{d \in \bZ}  \; \sum_{c \ne 0}  \frac{1}{(c z + d)^2}
$$
are absolutely convergent. By considering the difference of double sums,
we conclude that 
$$
\sum_{d \in \bZ}  \; \sum_{c \ne 0}
 \frac{1}{(c z + d) (c z + d + 1)}
$$
also has the property that the inner and outer summations are absolutely convergent.
We can use the formula for this double sum given above to conclude the following:
\begin{eqnarray*}
\sum_{c \ne 0} \; \sum_{d \in \bZ} \, \frac{1}{(c z + d)^2}
&=&
\sum_{d \in \bZ}  \; \sum_{c \ne 0}\left( \frac{1}{(c z + d)^2}
-
 \frac{1}{(c z + d) (c z + d + 1)}
\right)
\\
&=&
\sum_{d \in \bZ}  \; \sum_{c \ne 0}
\frac{1}{(c z + d)^2}
-
\sum_{d \in \bZ}  \; \sum_{c \ne 0}
 \frac{1}{(c z + d) (c z + d + 1)}
 \\
&=&
 \frac{2 \pi i} {z} + 
\sum_{d \in \bZ}  \; \sum_{c \ne 0}
\frac{1}{(c z + d)^2}.
\end{eqnarray*}
In particular, for any given $z \in \h$, the double
sum 
$$
\sum_{c \ne 0} \; \sum_{d \in \bZ} \, \frac{1}{(c z + d)^2}
$$
is not absolutely convergent. The same conclusion holds for the double
sum defining~$G_2$.



%%%%%%%%%%%%%%%%%%%%


\section {Transformation properties}

Let $z \in \h$.
We can now connect $z^{-2} G_2(-1/z)$ to $G_2(z)$.
Observe
\begin{eqnarray*}
\frac{G_2 (-1/z) }{z^2}
&=&
2 \zeta(2) + \sum_{d \in \bZ} \; \sum_{c \ne 0}  \, \frac{1}{(c z + d)^2} 
\\
&=&
2 \zeta(2)
+
\sum_{c \ne 0} \; \sum_{d \in \bZ} \, \frac{1}{(c z + d)^2}
 -
 \frac{2 \pi i} {z}
 \\
&=&
G_2(z)
 -
 \frac{2 \pi i} {z}.
\end{eqnarray*}
This shows that $G_2(z)$ is not weight-2 invariant under $SL_2(\bZ)$.
In fact it behaves similarly to the nonholomorphic function $z \mapsto 1/\mathrm{Im}(z)$:

\begin{lemma}
Let
$
f(z) = 1 / \mathrm{Im}(z)
$.
Then $f(z + 1) = f(z)$ and
$$
\frac{f(-1/z)}{z^2} = f(z) - \frac{2 i}{z}.
$$
\end{lemma}

\begin{proof}
Clearly $f(z+1) = f(z)$. 
Observe that~$\mathrm{Im} (-1/z) = \mathrm{Im}(z) / (z \overline z)$. Thus
$$
\frac{f(-1/z)}{z^2} - f(z)  = \frac{z \overline z} { z^2 \, \mathrm{Im} (z)} - \frac{1}{\mathrm{Im}(z)}
=
\frac{\overline z - z} {z \, \mathrm{Im} (z)}
=
\frac{-\mathrm{2\,  Im}(z) i} {z \, \mathrm{Im} (z)}
=
- \frac{2 i}{z}.
$$
\end{proof}

\begin{corollary}
The function $G_2(z) -  \pi / \mathrm{Im} (z)$ is weight-2 invariant under $SL_2(\bZ)$.
\end{corollary}

\begin{corollary}
Suppose $\gamma  \in SL_2(\bZ)$ is of the form
$$\gamma = \begin{bmatrix}
a & b \\
c & d
\end{bmatrix}.$$
Then
$$
\left( G_2  {[\gamma]_2} \right) (z) =  G_2 (z) - \frac{2 \pi i c }{c z + d}
$$
\end{corollary}

\begin{proof}
Since $G_2(z) -  \pi / \mathrm{Im} (z)$ is weight-2 invariant, we just need
to calculate $f [\gamma]_2$ where $f(z) = \pi/\mathrm{Im}(z)$.
\begin{eqnarray*}
f [\gamma]_2 (z) - f (z)
&=&
\frac{1}{(c z + d)^2} \frac{\pi}{\mathrm{Im}(\gamma z)} - \frac{\pi}{\mathrm{Im}(z)}
\\
&=&
\frac{1}{(c z + d)^2} \frac{\pi |c z + d|^2}{\mathrm{Im}(z)} - \frac{\pi}{\mathrm{Im}(z)}
 \\
&=&
\frac{\pi}{\mathrm{Im}(z)}
\left(
 \frac{c\overline z + d} {cz+d} -  \frac{cz + d} {cz+d} 
 \right)
 \\
 &=&
 - \frac{2 \pi i c}{c z + d} \, .
\end{eqnarray*}

\end{proof}

\end{document}
