
\documentclass {amsart}
\usepackage{amsmath, amsthm, amsfonts, amssymb}
%\usepackage{tikz} \usetikzlibrary{positioning} % for commutative diagrams

%%%% Theorem-type MACROS

\theoremstyle{plain}

\newtheorem{proposition}{Proposition}
\newtheorem{theorem}[proposition]{Theorem}
\newtheorem{corollary}[proposition]{Corollary}
\newtheorem*{observe}{Observation}
\newtheorem{lemma}[proposition]{Lemma}

\theoremstyle{definition}
\newtheorem{definition}{Definition} 
\newtheorem{exercise}{Exercise} 
\newtheorem{axiom}{Axiom}
\newtheorem{declaration}[axiom]{Declaration}

\theoremstyle{remark}
\newtheorem{example}{Example}
\newtheorem*{remark}{Remark}
\newtheorem*{note}{Note}
\newtheorem*{assumption}{Background Assumption}


%%%% Character MACROS
\newcommand{\defeq}{\stackrel{\mathrm{def}}{\, = \,}}

\newcommand{\bN}{{\mathbb{N}}}
\newcommand{\bZ}{{\mathbb{Z}}}
\newcommand{\Z}{{\mathbb{Z}}}
\newcommand{\bR}{{\mathbb{R}}}
\newcommand{\bC}{{\mathbb{C}}}
\newcommand{\bQ}{{\mathbb{Q}}}
\newcommand{\bF}{{\mathbb{F}}}
\newcommand{\bA}{{\mathbb{A}}}
\newcommand{\bP}{{\mathbb{P}}}

\newcommand{\h}{{\mathcal{H}}}
\newcommand{\gtk}{G_{2,k}}





%%%%%%%%%%  END OF MACROS  %%%%%%%%%%%%

\begin{document} 

\title{Fourier series of $G_{2,k}$}

\author{Modular Form Study Group}

\date{September 2014, modified \today}

\maketitle

%%%%%%%%%%%%%%%%%%%%

The purpose of this note is to prove the formulae for the Fourier series of $G_{2,2}$ and $G_{2,4}$ given in the text. In fact, it is not much harder to find the Fourier series of $G_{2,k}$ for any $k$. The text proves the formula
\[
G_2(\tau) = 2\zeta(2)  - 8\pi^2 \sum_{n=1} \sigma(n) q^n.
\]
The series $G_{2,k}(\tau)$ is $G_2(\tau) - k G_2(k\tau)$. We note that $\zeta(2) = \frac{\pi^2}{6}$; together with the below, this is sufficient to compute the constant term in $G_{2,k}(\tau)$. We will show that
\[
G_{2,k}(\tau) = 2(1-k)\zeta(2) - 8\pi^2 \sum_{n=1} \bigg(\sum_{\substack{0 < d \mid n \\ k \nmid d}} d\bigg) q^n.
\]
From the definition of $\gtk$ and the fact that $q(k\tau) = q(\tau)^k$, we have
\[
\gtk(\tau) = 2(1-k)\zeta(2) - 8\pi^2 \bigg(\sum_{n=1} \sigma(n) q^n - \sum_{n=1} k\sigma(n) q^{kn}\bigg).
\]
Thus it suffices to show that
\[
\sum_{n=1} \sigma(n) q^n - \sum_{n=1} k\sigma(n) q^{kn} = \sum_{n=1} \bigg(\sum_{\substack{0 < d \mid n \\ k \nmid d}} d\bigg) q^n.
\]
The sums on the left combine to yield $\sum c_n q^n$ for some $c_n$. If $k \nmid n$, then certainly $c_n = \sigma(n)$, and both left and right sides of the equation agree. Now suppose $n = km$. The coefficient of $q^n$ on the left is $\sigma(km) - k\sigma(m)$. But
\begin{align*}
  k \sigma(m) &= \sum_{0 < d \mid m} kd \\
  &= \sum_{\substack{0< d' | n \\ k \mid d'}} d'.
\end{align*}
The claim now follows.

Given that $\gtk$ is supposed to be a newform for $\Gamma_0(k)$, one wonders if the Fourier series has a moduli-theoretic interpretation.
\end{document}
