\documentclass{article}

\usepackage{amsmath, amsthm, amssymb}
%\usepackage{mathrsfs}
%\usepackage[all]{xy}
%\usepackage{enumerate}
%\usepackage[colorlinks]{hyperref}
%\usepackage{cleveref}

\theoremstyle{plain}
\newtheorem{theorem}{Theorem}[section]
\newtheorem{corollary}[theorem]{Corollary}
\newtheorem{proposition}[theorem]{Proposition}
\newtheorem{lemma}[theorem]{Lemma}
\newtheorem{conjecture}[theorem]{Conjecture}

\theoremstyle{definition} 
\newtheorem{definition}[theorem]{Definition}

\theoremstyle{remark}
\newtheorem{remark}[theorem]{Remark} 
\newtheorem{example}[theorem]{Example}


\renewcommand{\emptyset}{\varnothing}
\newcommand{\too}{\longrightarrow}
\renewcommand{\phi}{\varphi}
\newcommand{\eps}{\varepsilon}
\newcommand{\Z}{\mathbb{Z}}
\newcommand{\Q}{\mathbb{Q}}
\newcommand{\R}{\mathbb{R}}
\newcommand{\C}{\mathbb{C}}

\newcommand{\h}{{\mathcal{H}}}

\newcommand{\defeq}{\stackrel{\mathrm{def}}{\, = \,}}

%%%%%%%%%%  END OF MACROS  %%%%%%%%%%%%

\begin{document} 

\title{Section 1.2: Congruence Subgroups}

\author{Modular Form Study Group}

\date{September 2014, modified \today}

\maketitle

%%%

\section{The Modular Group}

In this section the standard congruence subgroups are defined.

\begin{definition}
If a subgroup $\Gamma$ of $SL_2(\Z)$ contains
$\Gamma(N)$ for some~$N$, then $\Gamma$ is called a \emph{congruence subgroup}.
\end{definition}

\end{document}

