\documentclass{article}

\usepackage{amsmath, amsthm, amssymb}
%\usepackage{mathrsfs}
%\usepackage[all]{xy}
%\usepackage{enumerate}
%\usepackage[colorlinks]{hyperref}
%\usepackage{cleveref}

\theoremstyle{plain}
\newtheorem{theorem}{Theorem}[section]
\newtheorem{corollary}[theorem]{Corollary}
\newtheorem{proposition}[theorem]{Proposition}
\newtheorem{lemma}[theorem]{Lemma}
\newtheorem{conjecture}[theorem]{Conjecture}

\theoremstyle{definition} 
\newtheorem{definition}[theorem]{Definition}

\theoremstyle{remark}
\newtheorem{remark}[theorem]{Remark} 
\newtheorem{example}[theorem]{Example}


\renewcommand{\emptyset}{\varnothing}
\newcommand{\too}{\longrightarrow}
\renewcommand{\phi}{\varphi}
\newcommand{\eps}{\varepsilon}
\newcommand{\Z}{\mathbb{Z}}
\newcommand{\Q}{\mathbb{Q}}
\newcommand{\R}{\mathbb{R}}
\newcommand{\C}{\mathbb{C}}

\newcommand{\h}{{\mathcal{H}}}

\newcommand{\defeq}{\stackrel{\mathrm{def}}{\, = \,}}

%%%%%%%%%%  END OF MACROS  %%%%%%%%%%%%

\begin{document} 

\title{Section 1.3: Complex Tori.}

\author{Modular Form Study Group}

\date{September 2014, modified \today}

\maketitle

%%%

\section{Lattices}

Suppose $\omega_1, \omega_2$ and $\omega'_1, \omega_2'$ are two bases
in~$\R^2$. Then these bases can
be related by a unique element of  $\mathrm{GL}_2(\R)$ called the change of
basis matrix. 
Two bases $\omega_1, \omega_2$ and $\omega'_1, \omega_2'$ generate the same lattice if and only 
the change of bases matrix has integral coefficients and has determinant~$\pm 1$.

For applications to complex tori and complex elliptic curves, we identify $\R^2$ with~$\C$.
One immediate advantage of doing so is that we can simply specify a canonical orientation. More
specifically, we can specify a standard order for a given basis by requiring that
$\omega_1/\omega_2 \in \h$.

Suppose $\omega_1, \omega_2$ and $\omega'_1, \omega_2'$ are two bases
in~$\R^2$ (identified with $\C$), and if $\gamma$ is the change of basis matrix
then (using the linear fractional transformation action of $\mathrm{GL}_2(\R)$ on~$\C$)
we have
$$
\frac{\omega'_1} {\omega'_2} = \gamma \left( \frac{\omega_1} {\omega_2} \right).
$$
In particular, if both bases are in standard order, then the determinant of the change of basis matrix is positive.

We conclude that two basis listed in standard order generate the same lattice if and only if they are related to each other via a matrix in~$\mathrm{SL}_2(\Z)$. (See also Lemma~1.3.1 and Exercise~1.3.1).

Finally we note that if $\omega_1, \omega_2$ and $\omega'_1, \omega_2'$
are two bases in standard order, then the lattice generated by the second basis
is a sublattice of the lattice generated by the first basis if and only if
the change of basis matrix has integer coefficients, and the determinant of
this matrix gives the index of the sublattice. (So the index is always finite).


\section{Complex Tori and Isogenies}

A \emph{complex torus} is $\C / \Lambda$ where $\Lambda$ is a lattice in~$\C$.
We view a complex torus as a Riemann surface with a complex structure, and as
an abelian group.

Any holomorphic map $\phi\colon \C / \Lambda\to  \C / \Lambda'$ lifts to a holomorphism
$\tilde \phi\colon \C \to \C$. For any $\lambda \in \Lambda$, the function
$\tilde \phi (z + \lambda) - \tilde \phi (z)$ is a constant. It has value in $\Lambda'$ and is
independent of the choice of lifting.
So the derivative of~$\tilde \phi$ is $\Lambda$-periodic and must be constant by Liouville's
theorem. So $\tilde \phi$ is a linear function $z\mapsto m z + b$.
Since $\tilde \phi$ lifts $\phi\colon \C / \Lambda\to  \C / \Lambda'$, we have $m\Lambda \subseteq \Lambda'$.

If $\phi\colon \C / \Lambda\to  \C / \Lambda'$ maps zero to zero, we can choose $b=0$ in our lifting.
Thus any such function can be described by the rule $z \mapsto m z$ for some unique $m\in \C$
with the property that $m \Lambda \subseteq \Lambda'$. Observe that this is a group homomorphism. Conversely given $m \in \C$ with $m \Lambda \subseteq \Lambda'$, the resulting
map is a well-defined holomorphic homomorphism.

We also conclude that every holomorphic map $\phi\colon \C / \Lambda\to  \C / \Lambda'$
is the composition of a homomorphism and a translation (or, if you prefer, a translation composed with a homomorphism).

\begin{definition}
A nonzero holomorphic between complex tori is called an \emph{isogeny}.
\end{definition}

From the above discussion we see that each isogeny is associated to a unique $m\in\C^\times$
such that $m \Lambda \subseteq \Lambda'$.  Clearly the composition
of isogenies corresponds to multiplication of the associated elements of~$\C^\times$.
If the inclusion $m\Lambda \subseteq \Lambda'$ is proper, 
then any element of $\Lambda'$ not in $m\Lambda $
gives a nonzero element of the kernel. 
We can go further: \emph{an isogeny is a
group isomorphism if and only if $m\Lambda = \Lambda'$}.

TO DO: MORE ON ISOGENIES (KERNAL IS ALWAYS FINITE, GIVES DEGREE, ETC)


%%%%%

\section{Dual Isogenies}

TO DO

%%%%%

\section{Weil Pairing}

Suppose $\omega_1, \omega_2$ is a basis for the lattice~$\Lambda$. Assume the basis
is in standard order, so that $\omega_1/\omega_2 \in \h$. Then the superlattice of $N$-torsion points has basis $\omega_1 / N,  \omega_2/N$. 

Let $P$ and~$Q$ be two elements in the superlattice of~$N$-torsion points. Then there is a matrix with integral coordinates expressing $P$ and $Q$ in term of
the basis~$\omega_1 / N,  \omega_2/N$. Let $d$ be the determinant of this matrix. Then
the Weil pairing is the $N$th root of unity given by
$$
e_N(P, Q) = \left(e^{2\pi i}/N\right)^d.
$$

\begin{lemma}
The Weil pairing is independent of the choice of basis $\omega_1, \omega_2$
of~$\Lambda$ (using the standard order).
\end{lemma}

\begin{proof}
The change of basis matrix relating two such bases gives a change of basis matrix 
for~$\omega_1 / N,  \omega_2/N$. The determinate of this matrix is one. The matrix expressing $P$ in~$Q$
in terms of the changed basis therefore has the same determinate. 
\end{proof}




\begin{lemma}
The Weil pairing depends only on the image of $P$ and~$Q$ in $\C/\Lambda$.
\end{lemma}

\begin{remark}
Note, up to sign, we can think of $d$ as the index of the group generated 
by $P$ and $Q$ inside the group generated by~$\omega_1 / N,  \omega_2/N$.
\end{remark}


\begin{lemma}
The Weil pairing is bilinear.
For example:  $$e_N(P_1 + P_2, Q) = e_N(P_1, Q) e_N(P_2, Q).$$
\end{lemma}

\begin{proof}
This follows from linearity in each row of the determinant.
\end{proof}

\begin{lemma}
The Weil pairing is alternating:
$$e_N(P, Q) = e_N(Q, P)^{-1}.$$
\end{lemma}

\begin{proof}
Switching rows of a matrix changes the sign of its determinant.
\end{proof}

\begin{lemma}
The Weil pairing non degenerate: for any $P$, if 
$e_N(P, Q) = 1$ for all $Q$ then $P = 0$.
\end{lemma}

\begin{proof}
Choose $Q = \omega_1/N$ and then choose $Q = \omega_2/N$.
\end{proof}

\begin{lemma}
The Weil pairing is compatible with multiplication by~$d$. In other words,
the map $E[dN] \to E[dN]$ and the $d$ power map on roots of unity are compatible.
\end{lemma}

\begin{proof}
The matrix of $P$ and~$Q$ in terms of the designated $dN$-torsion basis is
exactly the same as the matrix of $dP$ and $dQ$ in terms of the associated $N$-torsion 
basis.
\end{proof}


\begin{lemma}
The Weil pairing is compatible with isomorphisms between complex tori.
\end{lemma}

\begin{proof}
Represent the isomorphism by a multiplication by~$m$ map.
\end{proof}

\begin{lemma}
The Weil pairing $e_N(P, Q)$ is a primitive $N$th root of unity if and only
if $P$ and $Q$ generate the group of $N$-torsion.
\end{lemma}

\begin{proof}
If $P$ and~$Q$ generate, then modulo~$N$, the matrix representing $P$ and~$Q$ is
invertible. Thus its determinant is relatively prime to~$N$. 

Conversely, if the Weil pairing is a primitive root of unity, then the determinant of the 
 the matrix representing $P$ and~$Q$ must be relatively prime to~$N$. So modulo~$N$
 this matrix is invertible. The inverse expresses the designated $N$-torsion basis
 in terms of $P$ and~$Q$ (in the $N$-torsion group).
\end{proof}

\end{document}

