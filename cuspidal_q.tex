
\documentclass {amsart}
\usepackage{amsmath, amsthm, amsfonts, amssymb}
%\usepackage{tikz} \usetikzlibrary{positioning} % for commutative diagrams

%%%% Theorem-type MACROS

\theoremstyle{plain}

\newtheorem{proposition}{Proposition}
\newtheorem{theorem}[proposition]{Theorem}
\newtheorem{corollary}[proposition]{Corollary}
\newtheorem*{observe}{Observation}
\newtheorem{lemma}[proposition]{Lemma}

\theoremstyle{definition}
\newtheorem{definition}{Definition} 
\newtheorem{exercise}{Exercise} 
\newtheorem{axiom}{Axiom}
\newtheorem{declaration}[axiom]{Declaration}

\theoremstyle{remark}
\newtheorem{example}{Example}
\newtheorem*{remark}{Remark}
\newtheorem*{note}{Note}
\newtheorem*{assumption}{Background Assumption}


%%%% Character MACROS
\newcommand{\defeq}{\stackrel{\mathrm{def}}{\, = \,}}

\newcommand{\bN}{{\mathbb{N}}}
\newcommand{\bZ}{{\mathbb{Z}}}
\newcommand{\Z}{{\mathbb{Z}}}
\newcommand{\bR}{{\mathbb{R}}}
\newcommand{\bC}{{\mathbb{C}}}
\newcommand{\bQ}{{\mathbb{Q}}}
\newcommand{\bF}{{\mathbb{F}}}
\newcommand{\bA}{{\mathbb{A}}}
\newcommand{\bP}{{\mathbb{P}}}

\newcommand{\h}{{\mathcal{H}}}





%%%%%%%%%%  END OF MACROS  %%%%%%%%%%%%

\begin{document} 

\title{Cuspidal $q$ (DRAFT)}

\author{Modular Form Study Group}

\date{September 2014, modified \today}

\maketitle

%%%%%%%%%%%%%%%%%%%%

If the $q$ coordinate of point is small with respect to one cusp, how does that point look from the point of view of other cusps? In other words, how close to $1$ is the $q$-value with respect to other cusps?

\section{Draft}




Fix positive bounds $X$ and $Y$.
Let $S$ be the strip~$\{ x + y i \in \h \; \mid \; |x| \le X, \; y \ge Y \}$.
Let
$$ \alpha = \begin{bmatrix}
       a & b         \\
       c & d
 \end{bmatrix}  \in SL (2, \bZ).$$
 be such that $c \ne 0$. In other words, $\alpha$ transforms
 the cusp at infinity to the rational number~$a/c$.

Fix $z \in S$ and define $x, y \in\bR$ and $q \in \bC$ by the formulas $z = x + y i$
and $q = \exp (2 \pi i z)$.
 Let $z' = \alpha(z)$, and define $x, y \in \bR$ and $q \in \bC$ by the formulas $z' = x '+ y' i$
and $q' = \exp (2 \pi i z')$.

Clearly when $|q|$ is close to zero, $|q'|$ will be close to one.  We want to quantify this behavior. In particular, for what follows we will need lower bounds for $1 - |q'|$ in terms of~$|q|$.
The constants below are to be understood to depend on $X, Y, \alpha$, but to be independent
of the choice of~$z \in S$.

\begin{lemma}\label{l1}
There are positive constants $C_1$ and $C_2$ such that
$$
C_1 \; y \le |c z + d| \le C_2\; y.
$$
\end{lemma}

\begin{proof}
The imaginary part of $c z + d$ is $c y$, so the left inequality holds.
By considering real and imaginary parts we have (by the triangle inequality)
$$
|c z + d | \le |c x + d| + |c y| \le |c| X + |d| + |c| y.
$$
Let $K = |c| X + |d|$. So
$$
|c z + d | \le K + |c| y \le K \frac{y}{Y} + |c| y  = \left( \frac{K}{Y} + |c|\right) y
$$
\end{proof}

Clearly as $y$ goes to infinity, $y'$ approaches zero. The following quantifies how fast $y'$ approaches zero.

\begin{corollary}\label{c2}
There are constants $D_1$ and $D_2$ such that
$$
\frac{D_1}{y} \le y' \le \frac{D_2}{y}.
$$
\end{corollary}

\begin{proof}
This follows from the above lemma and the formula $y' = y / |c z + d|^2$.
In fact we can take $D_1$ to be $1/C^{2}_2$, and $D_2$ to be $1/C^2_1$
where $C_1$ and $C_2$ are bounds appropriate in the previous lemma.
\end{proof}

\begin{lemma}\label{l3}
There is a  constants $C$  such that
$$
|q'| \le 1 - C y'.
$$
\end{lemma}

\begin{proof}
Observe that $|q'| = e^{-2\pi y'}$. 
Since $y \ge Y$, the above corollary gives a bound $Y'$ such that $y' \le Y'$.
By the following lemma there is a positive constant $C$, depending on~$Y'$, such that
$$
e^{-2 \pi y'} \le 1 - C y'.
$$
\end{proof}

\begin{lemma}
Fix $t_0>0$. Then
for all $t \in [-t_0, 0]$
$$
e^t \le 1 + \frac{1 - e^{-t_0}}{t_0} \cdot t.
$$
\end{lemma}
\begin{proof}
Consider the line
connecting  $(-t_0, e^{-t_0})$ and $(0, 1)$.
Use the fact that the second derivative of $t \mapsto e^t$ is positive. 
\end{proof}

Here is our main proposition on the size of $q'$:

\begin{proposition}
There is a positive constant $C$ such that
$$
1 - \left| q' \right| \ge \frac{C}{\log \left|q^{-1}\right| }.
$$
\end{proposition}

\begin{proof}
Combining Corollary~\ref{c2} and Lemma~\ref{l3}, we can find a constant $K$ such that
$$
|q'| \le 1 -  \frac{K}{y}.
$$
However, $\log \left|q^{-1}\right| = 2 \pi y$.
So 
$$
|q'| \le 1 -  \frac{2 \pi K}{\log \left|q^{-1}\right| }.
$$
\end{proof}

We will also need the following:

\begin{proposition}
There are positive constants $D_1$ and $D_2$ such that
$$
 \frac{D_1}{\log \left|q^{-1}\right| } \le |c z + d| \le \frac{D_2}{\log \left|q^{-1}\right| }.
$$
\end{proposition}

\begin{proof}
This is just a restatement of Lemma~1 using the equation 
$\log \left|q^{-1}\right| = 2 \pi y$.
\end{proof}




\end{document}
