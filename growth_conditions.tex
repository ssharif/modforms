
\documentclass {amsart}
\usepackage{amsmath, amsthm, amsfonts, amssymb}
%\usepackage{tikz} \usetikzlibrary{positioning} % for commutative diagrams

%%%% Theorem-type MACROS

\theoremstyle{plain}

\newtheorem{proposition}{Proposition}
\newtheorem{theorem}[proposition]{Theorem}
\newtheorem{corollary}[proposition]{Corollary}
\newtheorem*{observe}{Observation}
\newtheorem{lemma}[proposition]{Lemma}

\theoremstyle{definition}
\newtheorem{definition}{Definition} 
\newtheorem{exercise}{Exercise} 
\newtheorem{axiom}{Axiom}
\newtheorem{declaration}[axiom]{Declaration}

\theoremstyle{remark}
\newtheorem{example}{Example}
\newtheorem*{remark}{Remark}
\newtheorem*{note}{Note}
\newtheorem*{assumption}{Background Assumption}


%%%% Character MACROS
\newcommand{\defeq}{\stackrel{\mathrm{def}}{\, = \,}}

\newcommand{\bN}{{\mathbb{N}}}
\newcommand{\bZ}{{\mathbb{Z}}}
\newcommand{\Z}{{\mathbb{Z}}}
\newcommand{\bR}{{\mathbb{R}}}
\newcommand{\bC}{{\mathbb{C}}}
\newcommand{\bQ}{{\mathbb{Q}}}
\newcommand{\bF}{{\mathbb{F}}}
\newcommand{\bA}{{\mathbb{A}}}
\newcommand{\bP}{{\mathbb{P}}}

\newcommand{\h}{{\mathcal{H}}}





%%%%%%%%%%  END OF MACROS  %%%%%%%%%%%%

\begin{document} 

\title{A Condition for Being Holomorphic at All Cusps}

\author{Modular Form Study Group}

\date{August 2014, modified \today}

\maketitle

%%%%%%%%%%%%%%%%%%%%

\section {Introduction}

In this document we elaborate on Proposition~1.2.4 of Diamond, Shurman (and
the companion Exercise 1.2.6). 
In particular we prove that if the Fourier coefficients of a candidate modular form (for some modular group~$\Gamma$) has polynomial bounds, then it is holomorphic at all of the cusps. In other words, it is a modular form for~$\Gamma$.

%%%%%%%%%%%%%%%%%%%%

\section {Removable singularities at infinity}

The key to the above result is to bound the growth of our candidate function as $z$
approaches the cusps. By transforming the cusp to infinity, we prove our result by
bounding the growth of the transformed function $f$ as $z$ goes to infinity. In this section we show that if $f$ is bounded
by an expression of the form $A y^r$, then $f$ is indeed holomorphic at infinity. In later sections
we will show that the transformed candidate has such growth for all cusps.

Let $f(z)$ be a periodic function on~$\h$ (or some region of $\bC$ containing all $z = x + yi$
with $y$ sufficiently large). Assume the period is a positive real number~$N$.
Then $f(z)$ has a Fourier expansion (Ahlfors Chapter 7, Section 1). In other words, $f(z) = g(q(z))$
where $q (z) = e^{2 \pi i z / N}$, and where
$$
g(q) = \sum_{n\in\bZ} a_n q^n
$$
is the Fourier expansion of~$f$.
The function $f$ is said to be holomorphic at infinity if and only if $g(q)$ is holomorphic
at zero. This occurs if and only if $q g(q)$ approaches zero as $q$ approaches zero.
In other words, $f$ is holomorphic at infinity if and only if $q(z) f(z)$ approaches zero as $z$ approaches infinity.
(See Theorem~7 of Ahlfors Chapter 4, Section 3).

In what follows we will consider functions $f$ such that for $z = x + yi$ with $y \ge B$ (where $B$ is some fixed bound),
$$
|f(z)| \le A y^t
$$
for some constant~$A$ and nonnegative integer~$t$.
In this case 
$$
|f(z) q(z)| \le A \frac{y^t} { e^{ 2 \pi y / N}}.
$$
The limit of ${y^t} / { e^{ 2 \pi y / N}}$ is zero as $y$ goes to infinity (L'Hopital's Rule). So we conclude that
\emph{since $f(z)$ is so bounded by $A y^t$, then it must be holomorphic at infinity}.


%%%%

\section {Comparing $\sum f(n)$ to $\int f(x) dx$}

We want to begin our investigation by translating a growth condition on Fourier coefficients to
a growth condition of $f$ near its cusps. This can be done via the familiar
trick of comparing a sum $\sum f(n)$ to the associated integral~$\int f(x) dx$.
We will describe how to get bounds in the case where $f$ has a single maximum.

Suppose that $f(x)$ is a nonnegative continuous function on $[0, \infty)$ that is 
increasing on $[0, m]$ and decreasing on~$[m, \infty)$. Assume also
that 
$$
\int_{0}^\infty f(x) dx \; < \; \infty.
$$
Then we have the following upper bound:
$$
\sum_{n=0}^\infty f(n) \le f(m) + \int_{0}^\infty f(x) dx.
$$

To establish this bound, let $M$ be a positive integer chosen so that $m$ is in the interval $[M - 1, M]$.
If $0 \le i \le M -2$ is an integer, consider the rectangle $A_i$ with base $[i, i + 1]$
and height~$f(i)$.
If $i \ge M + 1$ is an integer, then consider the rectangle $B_i$ with base $[i - 1, i]$
and height $f(i)$. 
Finally, let $C$ be the rectangle with base $[M, M+1]$ and height the minimum
of $f(M)$ and $f(M+1)$.

Note that the area of the union of these rectangles is equal to the integral of a step
function that is bounded above by $f$. So the sum of the areas of these rectangles
is bounded above by~$\int_{0}^\infty f(x) dx$. This gives us
$$
\sum_{n \in I} f(n) \le \int_{0}^\infty f(x) dx
$$
where $I$ is $\bN$ with one element (either $M$ or $M+1$) removed.
The result follows.

%%%

\section {Example: bounding $\sum n^r e^{-\lambda n}$}

We are particularly interested in the function $f(x) = x^r e^{-\lambda x}$ where $r \in \bN$
and $\lambda$ is a positive real constant. Note that for large $x$, $f(x) \le e^{-\lambda x / 2}$
so, 
$$
\int_{0}^\infty f(x) dx \; < \; \infty.
$$
Also, note that $f$ is increasing on $[0, r/\lambda]$ and decreasing on~$[r/\lambda, \infty)$.
So the above result gives the estimate
$$
\sum_{n=0}^\infty n^r e^{-\lambda n} \le (r/\lambda)^r e^{-r} + \int_{0}^\infty  x^r e^{-\lambda x} dx.
$$
With a simple change of variables of the integral ($u = \lambda x$), we can rewrite this
inequality as
$$
\sum_{n=0}^\infty n^r e^{-\lambda n} \le \frac{C_1}{\lambda^r} + \frac{C_2}{\lambda^{r+1}}
$$
where $C_1$ and $C_2$ are positive constants dependent on~$r$, but not on~$k$.

If we fix a bound $B >0$, then we can split the inequality into two simpler inequalities
(where $C_1, C_2$ are possibly different positive constants that depend on $B$ and $r$).
If $\lambda \ge B$ we have
$$
\sum_{n=0}^\infty n^r e^{-\lambda n} \le \frac{C_1}{\lambda^r} .
$$
If $0 < \lambda \le B$ we have
$$
\sum_{n=0}^\infty n^r e^{-\lambda n} \le \frac{C_2}{\lambda^{r+1}}.
$$
Observe that in the second case, the case that we will be needed in what follows,
the integral gives the estimate of the discrete sum.


\section{Alternative proof of bound on $\sum n^r e^{-\lambda n}$}
\label{sec:altern-proof-bound}

The sum $\sum n^r e^{-\lambda n}$ can be evaluated exactly, which leads to an alternative argument for the bound shown in the previous section. Let $L$ denote the differential operator $x \frac{d}{dx}$. Let $f(x) = \dfrac{1}{1-x}$, and let $f_r = L^r f$. We observe that
\[
\sum n^r e^{-\lambda n} = f_r(e^{-\lambda}).
\]
In principle, one can thus compute each sum explicitly; however, there is no obvious formula for the $f_r$. But we can come up with a sufficiently good description to bound the summation.
\begin{lemma}
  If $r \geq 1$, then there exists $p_r(x) \in \Z[x]$ of degree $\leq r$ such that $p_r(0) = 0$, $p'_r(0) = 1$, and
  \[
  f_r(x) = \frac{p_r(x)}{(1-x)^{r+1}}.
  \]
\end{lemma}

\begin{proof}
  Set $p_0(x) = 1$. Via the product rule, one checks that if $p_r(x)$ is defined by 
  \[
  p_r(x) = (1-x)^{r+1} f_r(x),
  \]
  then the recursion
  \[
  p_r = x(1-x)p'_{r-1} + rxp_{r-1}
  \]
  holds for $r \geq 1$. The claim follows from an induction argument.
\end{proof}
The induction can also show that $p_r$ is monic of degree $r$, but this conclusion is not necessary for the below.

\begin{proposition}
  For $\lambda$ a sufficiently small positive number, there exists a constant $C$ such that
  \[
  \sum n^r e^{-\lambda n} < \frac{C}{\lambda^{r+1}}.
  \]
\end{proposition}

\begin{proof}
  We use the fact that the summation equals $f_r(e^{-\lambda})$. Let $p_r(x)$ be as in the last lemma. Note that $p_r(x) = x + \cdots$, where the dots indicate higher order terms. Then $p_r(e^{-\lambda}) < k e^{-\lambda} < k$ for some $k > 0$ and sufficiently small $\lambda$. (Here the $k$ depends on $p_r$, and hence on $r$.) By Taylor's Theorem, $1-e^{-\lambda} > \ell \lambda$ for some $\ell > 0$ and small enough $\lambda$. The claim now follows.
\end{proof}

%%%

\section {Application to $q$-expansions}

Suppose that $f \colon \h \to \bC$ is a holomorphic function with real period~$N$.
Suppose that its $q$-expansion has the form
$$
\sum_{n=0}^\infty a_n  q^n
$$
where $q = e^{2 \pi i z  / N}$ and where for positive~$n$
$$
|a_n| \le C n^r
$$
for some real constant~$C$ and some nonnegative integer~$r$.

From this we get the estimate
$$
|f(z)| \le |a_0| + C \sum_{n=0}^\infty  n^r  \left|q^n \right| \le 
|a_0| + C \sum_{n=0}^\infty  n^r  e^{- (2 \pi y / N) n}
$$
where $z = x + y i$.

So if we assume $z$ is such that $y \le B$ for some bound~$B$, the estimate of the
previous section can be applied. It gives us the estimate
$$
|f(z)|  \le \frac{C_0}{y^{r+1}}.
$$
for some constant $C_0$ independent of~$z$.
(The constant $|a_0|$ gets absorbed by choosing $C_0$ sufficiently large.
So $C_0$ does depends on $C$, $r$, $B$, and $|a_0|$).




%%%

\section {Behavior of Growth Conditions at Cusps under Transformation}

Let $f: \h \to \bC$ be a holomorphic function on the upper half plane that is
invariant under the weight $k$ action of some modular group~$\Gamma$.
In other words, $f [\alpha]_k = f$ for all $\alpha \in\Gamma$. 

Let $a/c$ be a cusp
of~$f$. 
Choose $\alpha \in SL(2, \bZ)$ so that $\alpha$ has form
$$ \alpha = \begin{bmatrix}
       a & b         \\
       c & d
 \end{bmatrix}  \in SL (2, \bZ).$$
 (This is possible since $a$ and $c$ are relatively prime).
 Then $f [\alpha]_k$ is not necessarily equal to~$f$, but it is periodic with
 some positive integer period. Furthermore, by definition $f$ is holomorphic at 
 the cusp~$a/c$ if and only if $f [\alpha]_k$ is holomorphic at infinity.

We want to investigate how growth conditions at cusps translates into growth conditions
of $f [\alpha]_k$ at infinity. So  assume that there is a bound $B$, and constants $C_0 \ge 0$
and $s \in \bN$ so that
$$
|f(z)|  \le \frac{C_0}{y^{s}}
$$
for all $z = x + yi$ with $y \le B$.
As we will see, this translates into bounds for $f [\alpha]_k$.


First we consider the behavior of $f(\alpha(z))$ near~$\infty$. Write $z = x + y i$
and $\alpha(z) = x' + y' i$.
By continuity, there is a bound $B_0 > 0$ such that if $y \ge B_0$ then $y' \le B$.
If $y \ge B_0$ then
$$
|f(\alpha(z))|  \; \le \; \frac{C_0}{(y')^{s}}.
$$
However, we know that $y' = y / |c z + d|^2$, so
$$
|f(\alpha(z))|  \; \le \; C_0  \frac{ |c z + d|^{2 s}}{(y)^{s}}.
$$
By definition, $f [\alpha]_k (z) = f(\alpha (z)) (c z + d)^{-k}$.
Thus
$$
\left| f [\alpha]_k (z) \right|  \; \le \; C_0  \frac{ |c z + d|^{2 s - k}}{(y)^{s}}.
$$
Since $ f [\alpha]_k $ is periodic with rea period, we can assume~$|x|$
is bounded and so, if $y \ge B_0$,
$$|c| |y| \le |c z + d| \le |c x + d| + |c y| \le D |y|$$
for some constant~$D$ independent of~$y$. 

Thus, if $y \ge B_0$, 
$$
\left| f [\alpha]_k (z) \right|  \; \le \; A  y ^{s-k}
$$
for some constant~$A$ (independent of~$y$).

%%%

\section {Proof of main theorem}

We start with a candidate modular form $f$ 
of weight $k$ for a modular group~$\Gamma$.
In other words, 
$f: \h \to \bC$ is a holomorphic function on the upper half plane 
such that $f [\alpha]_k = f$ for all $\alpha \in\Gamma$. 

Suppose that the Fourier coefficients $a_n$ of $f$ (with $n\ge 1$) have 
bounds of the form.
$$
|a_n| \le C n^r.
$$
We saw about how this translates into bounds for~$f$ of the form
$$
|f(z)| \le \frac{C_0}{y^{r+1}}
$$
for $y$ sufficiently small. So this behavior applies when we are sufficiently
close to a cups $a/c$.  If $\alpha \in SL(2, \bZ)$ maps $a/c$ to $\infty$
we saw that $f [\alpha]_k$ is bounded (for $y$ sufficiently large) by bounds of the form
$$
|f [\alpha]_k (z)| \le A y^t
$$
where (without loss of generality) we assume $t$ is a nonnegative integer.

We saw above that this implies, in turn, that $f [\alpha]_k$ has a removable singularity at infinity.
In other words, $f$ is holomorphic at the cusp $a/c$.



%%%

\section {Other Notes}

This proof follows the outline on page 22, but of course with slightly different notation.
I believe the proof on page 22 - 23 needs a correction.
I believe on page 22 the conclusion of 1.2.6 (a) should be
$$
|f(\tau)| \le C_0 + C/ y^{r+1}
$$
(not $y^r$) as $y$ approaches~$0$ (not as $y$ approaches $\infty$).

Also I would use the fact that exponentials dominate polynomials (not that polynomials dominate logarithms).

The book promises a converse in Section 5.9.



\end{document}
