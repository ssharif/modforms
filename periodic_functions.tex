
\documentclass {amsart}
\usepackage{amsmath, amsthm, amsfonts, amssymb}
%\usepackage{tikz} \usetikzlibrary{positioning} % for commutative diagrams

%%%% Theorem-type MACROS

\theoremstyle{plain}

\newtheorem{proposition}{Proposition}
\newtheorem{theorem}[proposition]{Theorem}
\newtheorem{corollary}[proposition]{Corollary}
\newtheorem*{observe}{Observation}
\newtheorem{lemma}[proposition]{Lemma}

\theoremstyle{definition}
\newtheorem{definition}{Definition} 
\newtheorem{exercise}{Exercise} 
\newtheorem{axiom}{Axiom}
\newtheorem{declaration}[axiom]{Declaration}

\theoremstyle{remark}
\newtheorem{example}{Example}
\newtheorem*{remark}{Remark}
\newtheorem*{note}{Note}
\newtheorem*{assumption}{Background Assumption}


%%%% Character MACROS
\newcommand{\defeq}{\stackrel{\mathrm{def}}{\, = \,}}

\newcommand{\bN}{{\mathbb{N}}}
\newcommand{\bZ}{{\mathbb{Z}}}
\newcommand{\bR}{{\mathbb{R}}}
\newcommand{\bC}{{\mathbb{C}}}
\newcommand{\bQ}{{\mathbb{Q}}}
\newcommand{\bF}{{\mathbb{F}}}
\newcommand{\bA}{{\mathbb{A}}}
\newcommand{\bP}{{\mathbb{P}}}

\newcommand{\h}{{\mathcal{H}}}





%%%%%%%%%%  END OF MACROS  %%%%%%%%%%%%

\begin{document} 

\title{Periodic Functions}

\author{Modular Form Study Group}

\date{August 2014, modified \today}

\maketitle

%%%%%%%%%%%%%%%%%%%%

\section {Introduction}

In this document we consider periodic meromorphic functions on~$\bC$. For convenience, we will limit our attention to functions with period~$1$.  In fact, most of the functions we study will have poles set equal to~$\bZ$.

We will derive formulas for $\sum 1/(z - n)^k$, $\sin^{-2} (\pi z)$ and $\cot (\pi z)$. 
These in turn can be used to derive formulas for $\zeta(k)$ for even positive integer~$k$. 
See Ahlfors Chapter 5, Section 2 for more information.

This overlaps with some of Section 1.1 
in Diamond-Shurman,
particularly Exercises 1.1.5 and 1.1.7.

%%%%%%%%%%%%%%%%%%%%

\section {Fourier Expansions}


Let $f(z)$ be a periodic meromorphic function on~$\bC$ with period~$1$.
Because $f(z)$ is periodic, we can express $f(z)$ as $g(q(z))$ for some
meromorphic function $g$, where $q(z) = e^{2\pi i z}$.

By looking at the Laurent expansion of $g(q)$ we can write 
$$
f(z)  = \sum_{n \in \bZ} a_n e^{2\pi i n z}.
$$
which converges for $z = x + yi$ such that $y > B$ for some suitably chosen~$B$ (taken large enough to avoid the poles of~$f$).
This is called the \emph{Fourier expansion} of~$f$
and the coefficients $a_n$ are called the \emph{Fourier coefficients} of~$f$.
(See Ahlfors, Ch.~7, Sect.~1 for more information).

If $a_n = 0$ for all $n < 0$ we say that \emph{$f$ is holomorphic at~$\infty$},
and $a_0$ is the considered the value of $f$ at $\infty$.
A sufficient condition for $f$ to be holomorphic at~$\infty$
is for the values of~$f$ to be bounded in a region where $z = x + i y$ and $y \ge B$ for some
real~$B$.
In other words, it is sufficient for $g$ to be bounded in a neighborhood of~$q=0$.
A necessary and sufficient condition is that $q g(q)$ approaches zero as $q$ approaches zero.
(See Theorem~7 of Ahlfors Chapter 4, Section 3).

In the cases we are interested in, the Fourier expansion comes from the power series of a rational function in~$q$. We are mainly interested in $\cot \pi z$ and $\sin^{-2} (\pi z)$.
In these cases, we get everything we need from the
identities $\cos z = (e^{iz} + e^{-iz})/2$ and $\sin z = (e^{iz} - e^{-iz})/(2 i)$.
From these formulas and a bit of algebra we get
$$
\cot \pi z = i \, \frac{q + 1}{q - 1}
$$
and
$$
\sin^{-2} (\pi z) = - \frac{4 q}{(q - 1)^2}
$$
Note these expressions give the locations of the zeros and poles of these functions, the
Fourier expansions, and
the value of the functions at~$\infty$.

Other trig functions can be handled in a similar manner.
$$
\cos^{-2} (\pi z) = \frac{4 q}{(q + 1)^2}
$$
$$
\sin(2 \pi z) = \frac{q^2 - 1}{2 i q}
$$
$$
\cos (2 \pi z) =\frac{q^2 + 1}{2 q}
$$
$$
\tan (\pi z) = - i \frac{q - 1}{q + 1}
$$
and so on. Also various trig identities can be derived from these rational expressions.
For example: $1 + \tan^2 (z) = \sec^2(z)$.




%%%%%%%%%%%%%%%%%%%%

\section {The Function $\sum 1/ (z - d)^k$}

Now let $f(z) = \sum_{d \in\bZ}  1/ (z - d)^k$ where $k \ge 2$ is an integer.
Later we will consider the  more delicate case $k=1$.

First we consider convergence questions. Fix a positive bound~$X$, and
let $S$ be the set of $z  = x  + y i$ such that $|x| \le X$. Consider
the function $f_0 (z) =  \sum' 1/ (z - d)^k$ where $\sum'$ indicates that
we remove a finite number of terms. More specifically, we only consider
terms where $|d| \ge 2 X$. For such $d$, and $z \in S$,
$$
|d - z| \ge |d| - |z| \ge |d| - X \ge \frac{1}{2} |d|.
$$
Thus
$$
\sum_{|d| \ge 2 X} \left| \frac{1}{(z-d)^k} \right| \le \sum_{|d| \ge 2 X} \frac{2^k}{|d|^k}.
$$
This shows that the series associated with $f_0(z)$ is absolutely convergent, and converges
uniformly on~$S$. Thus $f_0(z)$ is holomorphic on~$S$.

We get $f(z)$ by adding a finite number of well-understood meromorphic functions
to the holomorphic $f_0(z)$.
This implies that $f(z)$ is meromorphic on~$S$ with poles only at the integers in~$S$,
and that the associated series is absolutely convergent when $z$ is not an integer.
Since $X$ was arbitrary, we extend these properties to the limiting case of~$S = \bC$.

Because of absolute convergence, we see that $f(z)$ is periodic with period~$1$.
So we can study the behavior of $f(z)$ by limiting our attention to the set $S$ defined
above with $X = 1/2$. On $S$ the above estimates gives us the estimate
$$
|f(z)| \le 2^{k+1} \zeta (k) + \frac{1}{|z|^k} \le 2^{k+1} \zeta (k) + \frac{1}{|y|^k}.
$$
This shows that $f(z)$ is holomorphic at infinity.

\subsection{The case~$k=2$.}  When $k=2$
the above considerations show that $f(z) - 1/z^2$ is holomorphic
in a neighborhood of $z=0$. Note also that $\pi^2 / \sin^{2} (\pi z) - 1/z^2$ is likewise
holomorphic in a neighborhood of $z=0$. Thus the difference
$$f(z) - \frac {\pi^2} {\sin^{2} (\pi z)} $$
 is holomorphic in a neighborhood of~$z = 0$. By periodicity,
the difference is holomorphic at all integers, and hence holomorphic on all of~$\bC$. Above
we showed that $\sin^{-2}(\pi z)$ and $f(z)$ are holomorphic at~$\infty$, so the
difference is holomorphic on the extended complex plane.
This means the difference is a constant.
Since the difference is the difference of odd functions, it must be $0$. Thus
$$
  \sum_{d \in\bZ}  \frac{1}{ (z - d)^2} = \frac {\pi^2} {\sin^{2} (\pi z)}.
$$
Thus, since we already know the $q$-expansion of $\sin^{-2} (\pi z)$, we get the
$q$-expansion of $\sum_{d \in\bZ}  1/ (z - d)^2$.
(Note: in particular, we see that $f$ has value~$0$ at infinity. However, we can establish this
directly, for $k \ge 2$, by considering $S$ with $X$ large.)


%%%%%%%%%%%%%%%%%%%%

\section {The Period Function $\sum 1/ (z - d)$}

The case $k=1$ is a bit trickier because lack of absolute convergence.
In the definition we need to define how we take the limit of the sum. So define
the sum as follows:
$$
f (z)  = \lim_{N \to \infty} \sum_{-N}^{N} \frac{1}{z - d}.
$$
We will establish the convergence of the sum on the right. 
Note that  we can rewrite~$f$ as an absolutely convergent series in two
fairly natural ways:
$$
f(z) = \frac{1}{z} +  \sum_{d = 1}^\infty \left( \frac{1}{z - d} + \frac{1} {z + d} \right)
=
\frac{1}{z} +  \sum_{d = 1}^\infty \frac{2 z}{z^2 - d^2} 
$$
or
$$
f(z) =  \frac{1}{z}  + \sum_{d \ne 0}\left( \frac{1}{z - d} + \frac{1}{d} \right)
= \frac{1}{z}  + \sum_{d \ne 0}  \frac{z}{d(z - d)}
$$

To see the absolute convergence, we modify the argument employed for~$k > 1$ above.
Fix a bound $Z$ and let $S$ be the set of all $z \in\bC$ with $|z| \le Z$.
Consider
the function $f_0 (z) = \sum'  {z}/{d(z - d)}$ where $\sum'$ indicates that
we remove a finite number of terms. More specifically, we only consider
terms where $|d| \ge 2 Z$. For such $d$, and $z \in S$,
$$
|d - z| \ge |d| - |z| \ge |d| - Z \ge \frac{1}{2} |d|.
$$
Thus
$$
\sum_{|d| \ge 2 Z} \left| \frac{z}{d (z-d)} \right| \le \; Z \sum_{|d| \ge 2 X} \frac{2}{|d|^2}.
$$
This shows that the series associated with $f_0(z)$ is absolutely convergent, and converges
uniformly on~$S$. Thus $f_0(z)$ is holomorphic on~$S$.
(We get $|d  + z| \ge |d|/2$ as well, so we can get a similar estimate for the first expression
for $f(z)$).


We get $f(z)$ by adding a finite number of well-understood meromorphic functions
to the holomorphic $f_0(z)$.
This implies that $f(z)$ is meromorphic on~$S$ with poles only at the integers in~$S$,
and that the associated series is absolutely convergent when $z$ is not an integer.
Since $Z$ was arbitrary, we extend these properties to the limiting case of~$S = \bC$.

Since the series defining $f$ is absolutely convergent and is locally uniformly convergent,
we can obtain the derivate of $f$ by differentiating term by term.
Thus
$$
f'(z) =  - \frac{1}{z^2}  + \sum_{d \ne 0}  \frac{-1}{(z - d)^2}  = -  \sum_{d \in \bZ}  \frac{1}{(z - d)^2} .
$$
From our earlier result this implies
$$
f'(z) =  -\frac {\pi^2} {\sin^{2} (\pi z)}.
$$
In particular,
$$
f(z) = \pi \cot \pi z + C.
$$
for some constant~$C$. However $f$ and $\cot z$ are both odd functions. Thus
$C = 0$.


\section {Values of  $\zeta (k)$}

We can compute $\zeta (k)$ for even positive $k$ by finding the power series
of $\pi z \cot \pi z$ in two ways, based on the above identity.

Recall that 
$$
\cot \pi z = i \, \frac{q + 1}{q - 1} =  i  + \frac{2i}{q-1}
$$
where $q = e^{2 \pi z i}$.  

The Bernoulli numbers $B_k$ are defined by the terms of the power series
for $z / (e^z - 1)$.
$$
\frac{z}{e^z - 1} \;\defeq\; \sum_{k = 0}^\infty B_k \frac{z^k}{k!}
$$
So
$$
\frac{2 \pi zi }{q - 1} =  \frac{2 \pi zi }{e^{2 \pi z i}  - 1}  = 
\sum_{k = 0}^\infty B_k \frac{(2 \pi i)^k z^k}{k!}.
$$
In particular,
$$
\pi z \cot \pi z = \pi z i + \frac{2\pi z i}{q-1} = \pi z i  + \sum_{k = 0}^\infty B_k \frac{(2 \pi i)^k z^k}{k!}.
$$

On the other hand, above we showed
$$
 \pi z \cot \pi z = z f(z)
$$
where
$$
f (z)  = \lim_{N \to \infty} \sum_{d=-N}^{N} \frac{1}{z - d}.
$$
It is straightforward to find the higher derivatives of $g(z) = f(z) - 1/z$ at $z = 0$.
Note $g(0)$, and more generally $g^{(n)} (0) = 0$ for all even~$n$ since the $d=k$ and
$d = -k$ terms cancel. When $n$ is odd, the $d=k$ and
$d = -k$ terms are equal. Since the $k$th derivative of $(z-d)^{-1}$ is $-k! (z-d)^{-k-1}$ when
$k$ is odd, we get
$$
g^{(k)} (0)= - 2 k! \zeta (k+1)
$$
for odd~$k$. So the power series expansion of $g(z)$ is
$$
g(z) = \sum_{\textit{$k$ odd}} -2 \zeta(k+1) z^k.
$$
The power series expansion of $z f(z)$ is then
$$
z f(z) = 1 + \sum_{\textit{$k\ge 2$ even}} - 2 \zeta(k) z^k
$$
Thus
$$
1 + \sum_{\textit{$k\ge 2$ even}} - 2 \zeta(k) z^k = 
\pi z i  + \sum_{k = 0}^\infty B_k \frac{(2 \pi i)^k z^k}{k!}.
$$
From this we see that $B_0 = 1$, $B_1 = -1/2$ and that $B_k = 0$ for $k>1$ odd.
We also see that, for even $k \ge 2$,
$$
-2 \zeta(k) = \frac{(2 \pi i)^k}{k!} B_k = (-1)^{k/2} \frac{(2 \pi)^k}{k!} B_k.
$$
Thus $B_k$ is negative if $4 \mid k$, and positive for even $k$ with $4 \nmid 4$.

From the definition of $B_k$ we see that $B_2 = 1/6$, $B_4 = - 1/30$,
and $B_6 = 1/42$.
Thus
$$
\zeta (2) = \frac{\pi^2}{6}
$$
$$
\zeta (4)  =  \frac{\pi^4}{90}
$$
$$
\zeta (6) = \frac{2 \pi^6}{45} B_6 = \frac{\pi^6}{45\cdot 21} = \frac{\pi^6}{945}.
$$






\end{document}
